\section{Introduction}
Memory leak is a standard cause of errors in low-level programming languages
such as C and C++. Those languages provide manual memory management and require
explicit deallocation of structures. The category of errors can eventually
causes a system to run our of memory, and is difficult to identify. \\ \\

We describe a LLVM~\cite{llvm} implementation on an analysis which
can detect a set of errors caused by the memory leak. The implementation is
based on the algorithm proposed by Orlovich and Rugina~\cite{rugina}. \\ \\

In this implementation we analysis and detect the potential memory leak error
by checking whether a statement can cause a potential loss of reference to a 
heap cell. The definition of a leaked (eroor) cell is given~\cite{rugina} as: the 
program or the run-time system doesn't reclaim its memory when the lifetime 
has ended. \\ \\

The analysis works as a backward Dataflow analysis. For each statement that 
can cause a potential loss of heap cell reference, the analysis assume that 
it causes a loss of the last reference to the cell, and thus, an error cell
is formed. The analysis then release a detection probe to starts a backward 
Dataflow analysis from that programming point. If a contradiction is formed 
and the analysis reaches a bottom, the probe statement is proved to be safe. 
Otherwise, if the analysis reaches a top, or it reaches the program entry, 
a warning of potential memory leak is released. \\ \\

Section~\ref{s:algorithm} introduces the syntax, notations and algorithm
used by this implementation. \\ \\

The evaluation~\ref{s:evaluation} is done through benchmarking. A set of 6 kinds
of microbenchmarks is used to show the correctness on different categories of
statements being analyzed. Then, two C programs from the GNU C Library source
code~\cite{glibc} are used to evaluate the performance on larger programs. No 
false positive are shown in the two GNU C programs.
