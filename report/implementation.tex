\section{Implementation Details and Improvements}
\label{s:implementation}

The analysis uses a worklist-based Dataflow algorithm. 
This section describes some implementation details as well as some other 
constraints added in order to decrease some kinds of false positives.

\paragraph{The Representation of Pointers in the Hit Set}

One difference between the implementation and the paper's description is that 
instead of storing the pointer ${*e_0}$,
we allow the Hit set in the dataflow value to store either ${*e_0}$ or its 
address ${e_0}$ to represent the pointer. This is to adapt \llvm's SSA form 
representation of the store instructions, from which only the address of the 
pointer can be retrieved, instead of a concrete representation of a pointer itself. 
e.g. When a variable address is stored into a pointer, the SSA form 
representation will be:
\begin{align*}
store i32* \%1, i32** \%a, align 8
\end{align*}
where $\%a$ is the address of the pointer $a$.

Accordingly, changes are made into the transfer functions and the deduction 
rules to consider both the cases when either a pointer's value or a pointer's 
address is stored to represent the pointer's existence in the Hit set. 
Such changes also guarantee that the new rules are equivalent to the rules in 
the Rugina~\cite{rugina} paper. Take the deduction rule [SUBST 2] in 
Figure~\ref{fig:rule_ass} 
as an example, it is decomposed into two saparate rules in our new representation:
\begin{prooftree}
  \AxiomC{$e_0\#\omega$}
  \AxiomC{$\exists(*e')\in M$}
  \AxiomC{$e'=e_0$}
  \TrinaryInfC{$^-e_1$}
\end{prooftree}
and
\begin{prooftree}
  \AxiomC{$e_0\#\omega$}
  \AxiomC{$S\cap pt(e_0)=\emptyset$}
  \BinaryInfC{$^-e_1$}
\end{prooftree}


\paragraph{Branches, Worklist and Exit Conditions}

When there are branches in the program, the analysis reaches a $\bot$ if and only 
if contradiction exist in all branches. The Rugina~\cite{rugina} paper does not 
describe this part in details. So in this paragraph we will have a 
short discussion on the details of worklist, branches and exiting conditions.

Firstly, we added another boolean value $Contr$ into the flow value to 
represent a contradiction's being discovered. And the join operation is 
extended:
\begin{align*}
  (S_1,H_1,M_1,Contr_1) &\sqcup (S_2,H_2,M_2,Contr_2) \\
  = (S_1\cup S_2,H_1\cap H_2, & M_1\cap M_2,Contr_1\wedge Contr_2)
\end{align*}

When a probe statement is reached, the analysis starts a backward data flow 
by inserting into the worklist the basic block where 
the probe resides. Then, the analysis continues to work until one of the 
following conditions is met:
\begin{itemize}
  \item A $\top$ is reached: the analysis exit and prompt an warning of 
    potential memory leak;
  \item A $\bot$ is reached in the local basic block of the probe: 
    the probe is proved to be safe and the analysis exits without checking all 
    the branches;
  \item A $\bot$ is reached in the entry block from all of its successors:
    the probe is proved to be safe and the analysis exits;
  \item The entry point is reached and the worklist is empty.
\end{itemize}

If a basic block gets a $\bot$ through the meet operation on all its successors,
or if a new $\bot$ is reached inside a basic block, we will skip the analysis on 
the remaining statements in that block, and add its predecessors into the 
worklist.

Also, to avoid redundent loops for the convergence of the dataflow values, 
we only insert a predecessor $BB_p$ of a basic block $BB_s$ if and only if they 
meet with all the following conditions:
\begin{itemize}
  \item $BB_s$ has been changed;
  \item The $Contr$ in $BB_p$'s out value equals to $false$.
\end{itemize}


\paragraph{Improvements}

Besides the cases discussed in the Rugina~\cite{rugina} paper, we also exclude 
some other cases which can generate false positives;

\begin{itemize}
  \item The first assignment of an uninitialized pointer;
  \item The storage into a non-pointer type;
\end{itemize}
