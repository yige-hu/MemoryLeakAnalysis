\section{Algorithm}
\label{s:algorithm}

We use the algorithm proposed by Orlovich and Rugina~\cite{rugina}. It is a 
backward Dataflow analysis, validating a statement's safety by searching for 
contradictions. It releases a detection probe for each suspicious statement 
that might lead to a loss of the last reference to a heap cell, assuming that 
a memory leak happens. A backward Dataflow analysis is immediately started from
that program point to check whether a contradiction exists on all possible 
branches, to prove that the probe statement is safe. If the Dataflow analysis 
reaches a $\top$, or the program entry is reached, the analysis reports a 
potential memory leak.

A pointer analysis should be used to get the aliasing information, 
which is then used to get the disjointness information. Flow-insensitive 
pointer analysis such as Andersen style~\cite{andersen} or 
Steensgaard's~\cite{steensgaard} can be used to get two kinds of required 
information:

\begin{itemize}
  \item $Rgn$: the set of memory regions 
    (i.e. different regions model disjoint sets of memory locations);
  \item $pt(e)$:points-to information between regions.
\end{itemize}

In this work, we use Chen's implementation on the Andersen's 
analysis~\cite{chen} to get the above two information.
The regions are represented by the $AllocaInst$ or $CallInst$ to the $malloc()$ 
function in \llvm's SSA form. And the points-to set is represented by the 
set of the memory regions a pointer may point to.


\subsection{Language Semantics}
\label{ss:semantics}

In this section we describe the semantics of the core imperative language 
used by the algorithm, which helps better understand the later 
paragraphs~\ref{ss:dataflow}.

\paragraph{Syntax} for statements and expressions: \\

$Statements s\in St s::= *e_0\gets e_1 | *e\gets malloc | gree(e) | 
cond(e_0\equiv e_1) | return e | enter$ \\

$Expressions e\in E e::= n | a | *e | e.f | e_0 \oplus e_1 $ \\

where, \\
$n\in \mathbb{Z}$ ranges over numeric constants ($NULL=0$), \\
$a\in \mathbb{A}$ ranges over symbolic addresses, \\
$f\in \mathbb{F}$ over structure fields, \\
$\oplus$ over arithmetic operators, \\
$\equiv$ over the comparison operators $=$ and $\ne$.

\paragraph{Mem(e) Function} 

The set $Mem(e)$ denotes the subexpressions of $e$ that represent memory 
locations. This set is defined recursively as follows:

\begin{align*}
  Mem(n) &= Mem(a)=\emptyset \\
Mem(e.f) &= Mem(e) \\
 Mem(*e) &={*e}\cup Mem(e) \\
Mem(e_0 \oplus e_1) &= Mem(e_0)\cup Mem(e_1)
\end{align*}

%The corresponding function in the implementation is the $getMem(e)$ 
%function in \leakanalysis. \\ \\

Because of \llvm's SSA form, the expression context $\varepsilon$ described 
by the Rugina's paper~\cite{rugina} needn't be considered.

\paragraph{Disjointness}

The analysis uses the points-to information from the Andersen's analysis 
to resolve alias queries. An expression $e$ is disjoint from a region set 
$rs$, written $e\#rs$, if updates in any of the regions in $rs$ do not 
affect value of $e$:

\begin{align*}
e\#rs \ iff \ \forall(*e')\in Mem(e). pt(e')\cap rs=\emptyset
\end{align*}



\subsection{Dataflow Elements}
\label{ss:dataflow}

This subsection describes the Dataflow elements in the memory leak analysis. 

\paragraph{Direction}

It is a backward analysis.

\paragraph{Flow Value and Error Cell Abstraction}

The flow value as well as the error cell is modeled by a triple of the form 
$(S,H,M)$, where:

\begin{itemize}
  \item $S\subseteq Rgn$ is the conservative set of regions that might hold 
    pointers to the error cell;
  \item $H$ is the Hit set, a set of expressions that point to the error cell; and 
  \item $M$ is the Miss set, a set of expressions that do not reference the cell.
\end{itemize}

\paragraph{Detection Probes and Initial Flow Values}

The analysis issues leak probes for the statements that can cause a potential 
memory leakage. Especially, at the following programming points:

\begin{enumerate}[(a)]
  \item Assignments: for each $*x_0\gets x_1$ build a dataflow triple 
    $(pt(e_0),{*e_0},{e_1})$;
  \item Allocations: each $*x_0\gets malloc$ is probed as $*x_0\gets 0$;
  \item Deallocations: for each $free(e)$, issue probes that correspond to a 
    sequence of assignments $*(e.f)\gets 0$, for each field $f$ of $e$. This 
    checks for leaks caused by freeing a cell that holds the last reference 
    to another cell.
\end{enumerate}

This implementation does not deal with the last case, since from the 
\llvm IR got by clang, the information of the fields in a struct is hidden and 
cannot be accessed directly.

\paragraph{Join Operation}

The partial ordering for the flow value is such that 

\begin{align*}
  (S_1,H_1,M_1)\sqsubseteq (S_2,H_2,M_2) \ &iif \ 
  S_1\subseteq S_2, H_1\supseteq H_2 and M_1\supseteq M_2
\end{align*}

Therefore, the join operation is defined as:

\begin{align*}
  (S_1,H_1,M_1)\sqcup (S_2,H_2,M_2) &= (S_1\cup S_2,H_1\cap H_2,M_1\cap M_2)
\end{align*}


\paragraph{Transfer Function}

The transfer function differs on four categories of statements: assignment, 
allocation, deallocation and condition. A set of helper functions are used 
by the transfer functions:

\begin{align*}
  ImplicitMiss(e,(S,H,M)) &= (e=n) \vee (e=a) \vee (e=*n) \vee (e=n.f) \vee \\
                          &\ \ \ \ (e=*e' \wedge (S\cap pt(e')=\emptyset)) \\
  Miss(e,(S,H,M)) &= e\in M \vee ImplicitMiss(e,(S,H,M)) \\
Infeasible(S,H,M) &= \exists e \in H. Miss(e,(S,H,M)) \\
   Cleanup(S,H,M) &= (S,H,M'), \ where: \\
  M' &= {e|e\in M \wedge \urcorner ImplicitMiss(e,(S,H,M))} \\
\end{align*}

where $ImplicitMiss()$ generates new misses;
$Miss()$ checks whether an expression is in the Miss set or is a new miss; 
$Infeasible()$ indicates a contradiction;
$Cleanup()$ removes implicit miss to keep the dataflow facts as small as possible and avoid redundancy, while the correctness is conserved.

Four kinds of statements are analyzed: \\
(1) Assignments

\[
\Vert *e_0 \gets e_1 \Vert(S,H,M)=\begin{cases}
\bot& \text{if $Infeasible(S',H',M')$},\\
Cleanup(S',H',M')& \text{otherwise}.
\end{cases}
\]

where $S'=S\cup pt(e_0)$, and $H',M'$ are derived using the rules:

\begin{align*}
a
\end{align*}
