\section{Evaluation}
\label{s:evaluation}

We evaluate the correctness of the analysis using a set of six categories of 
microbenchmarks comprised of small program pieces, and a set of benchmarks 
from the GNU C Library source code~\cite{glibc}.

\subsection{Microbenchmarks}
\label{microbenchmark}

There are 6 sets of microbenchmarks as shown in Table~\ref{tbl:micro}, 
containing a total of 12 testcases, each of which is a small-scale C program.
The first 4 sets are used to check the correctness of the analysis on 4 
categories of statements being analyzed. The 5th is for evaluation on 
structure type. And the 6th is a realistic function for the linked-list.

%\begin{itemize}
%  \item 1\_assignment: assignment statements of form: $*x_0\gets x_1$;
%  \item 2\_malloc: allocation statements of form: $*x_0\gets malloc()$;
%  \item 3\_free: deallocation statements of form: $free(e)$;
%  \item 4\_cond: conditions of form: $cond(e_0\equiv e_1)$;
%  \item 5\_struct: analysis on fields in structure types;
%  \item 6\_data\_structure: a realistic function for linked-list.
%\end{itemize}


\begin{table}[t!]
  \centering
    %\setlength{\tabcolsep}{2pt}
    %\def\arraystretch{.8}
    \begin{tabular}{|l|l|}
    \hline
    Microbenchmark set & description\\
    \hline
    \hline
    1\_assignment & assignment statements of form: $*x_0\gets x_1$ \\
    \hline
    2\_malloc & allocation statements of form: $*x_0\gets malloc$ \\
    \hline
    3\_free & deallocation statements of form: $free(e)$ \\
    \hline
    4\_cond & conditions of form: $cond(e_0\equiv e_1)$ \\
    \hline
    5\_struct & analysis on fields in structure types \\
    \hline
    6\_data\_structure & a realistic function for linked-list \\
    \hline
    \end{tabular}
    \caption{\label{tbl:micro} Description on microbenchmarks}
\end{table}

The analysis discovers all potential memory leak for the above microbenchmarks 
with no false negative and false positive.


\subsection{Benchmarks}
\label{benchmarks}

The benchmarks is composed of two C programs taken from the GNU C Library 
source code~\cite{glibc}. For the convenience concern, the C programs selected 
are those can be compiled by clang as an independent C program without other
dependencies. They are two atest programs in the GNU C math package:

\begin{itemize}
  \item glibc\_math/atest-exp.c
  \item libc\_math/atest-sincos.c
\end{itemize}

No false positive are reported for the above two programs using our memory leak 
analysis.

Two other GNU C Libray programs used for cryptograph (in the folder \crypt) 
is also tested, but throws assertion errors in the Andersen's 
analysis implemented by Chen.~\cite{chen}.

Due to the time constraint, we have not found enough large pieces of standard C 
library codes that are suitable for the test.
